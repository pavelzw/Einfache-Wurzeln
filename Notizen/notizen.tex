\documentclass{article}

\IfFileExists{skript.sty}{\usepackage{skript}}{\usepackage{../skript}}

\begin{document}
\title{Einfache Wurzeln - Notizen}

\section{Wiederholung}

Spiegelung an Hyperebene \( \mathscr{P} \):
Sei \( 0 \neq r \in \mathscr{P}^\bot \).\\
Eine Spiegelung an \( \mathscr{P} \) lässt sich 
durch 
\[ s_r(x) = x - 2\frac{\scalarprod{x}{r}}{\scalarprod{r}{r}} r \]
beschreiben.\\
Wenn \( x \in \mathscr{P} \), dann gilt 
\( s_r(x) = x \). Außerdem gilt \( s_r(r) = -r \).
Da eine Basis \( B \subset \mathscr{P} \cup \set{r} \) 
existiert, folgt aus linearer Fortsetzung, dass 
\( s_r \) eine Spiegelung an \( \mathscr{P} \) 
darstellt.
Deshalb auch \( s_r^2 = 1 \). \\
\( \Rightarrow \pm r \) sind Wurzeln von \( s_r \).


\section{Positive und einfache Wurzeln}

Wurzelsystem \( \Phi \) kann im Vergleich zu 
\( \dim V \) sehr groß sein.\\
Also suchen wir nach linear unabhängigen Untermengen 
von \( \Phi \), mit der wir \( \Phi \) konstruieren 
können.\\
Jede Wurzel aus \( \Phi \) soll eine \gqq{einfache} 
Wurzel sein, d. h. sie sollen alle das gleiche Vorzeichen 
haben.

\begin{defi}[Ordnung im Vektorraum] %todo lambda und mu zu u, v ersetzen
    Transitive Relation auf \( V \times V \) soll 
    folgende Axiome erfüllen.
    \begin{enumerate}
        \item \( \forall \lambda, \mu \in V \) gilt 
        entweder \( \lambda > \mu, \lambda < \mu \) oder 
        \( \lambda = \mu \).
        \item \( \forall \lambda, \mu, \nu \in V \) gilt:
        \[ \mu < \nu \Rightarrow \mu + \lambda < \nu + \lambda. \]
        \item Seien \( \mu, \nu \in V, c \in \R \setminus \set{0} \).
        \[ \mu < \nu \Rightarrow 
        \begin{cases}
            c \mu < c \nu, & c > 0 \\
            c \mu > c \nu, & c < 0
        \end{cases}. \]
    \end{enumerate}
    Die Summe positiver Vektoren ist positiv, die skalare 
    Multiplikation mit positiven reellen Zahlen ist ebenfalls 
    positiv.
\end{defi}
Wie kommen wir auf eine Ordnung?
Wähle geordnete Basis \( B = (b_1, \ldots, b_n) \). 
Seien \( v = \sum_{i=1}^n \lambda_i b_i \) und 
\( w = \sum_{i=1}^n \mu_i b_i \).
Es gelte nun:
\[ v < w \Leftrightarrow
\lambda_k < \mu_k \text{ für kleinstes } k \text{ mit } 
\lambda_k \neq \mu_k. \]

\begin{defi}[Positives Wurzelsystem]
    \( \Pi \subset \Phi \) heißt positives System, wenn 
    es aus allen positiven Wurzeln aus \( \Phi \) besteht.\\
    \( \Rightarrow \Phi \) lässt sich aufteilen in 
    \( \Pi \) und \( -\Pi \) (negatives System), da 
    Wurzeln als Paare vorkommen.
    \[ \Pi \cap -\Pi = \emptyset, \Pi \cup -\Pi = \Phi. \]
\end{defi}

\begin{defi}[Einfaches System]
    \( \Delta \subset \Phi \) nennen wir ein einfaches System und 
    \( v \in \Delta \) einfache Wurzeln, wenn 
    \( \Delta \) eine Basis für \( \langle \Phi \rangle \) 
    ist. \\
    Außerdem muss für jede Wurzel gelten, dass sie 
    sich als Linearkombination mit ausschließlich 
    nichtnegativen oder nichtpositiven Faktoren 
    schreiben lässt.
\end{defi}
Existenz eines einfachen Systems nicht trivial! 
% beweis auf S. 8 Humphreys
\begin{bem}[Kardinalität einfacher Wurzelsysteme]
    Die Kardinalität jedes einfachen Wurzelsystems 
    ist eine Invariante von \( \Phi \), weil es die 
    Dimension von \( \langle \Phi \rangle \) misst.\\
    Wir nennen es \( \rk W \). (\( W \) ist die endliche 
    Spiegelungsgruppe).
\end{bem}
% todo einfache Systeme für Gruppen Beispiele

\section{Beziehungen zwischen verschiedenen 
Wurzelsystemen}

\begin{satz}
    Sei \( \Delta \) ein einfaches Wurzelsystem, welches 
    \( \Pi \) erzeugt. Falls \( \alpha \in \Delta \), 
    dann gilt \( s_\alpha(\Pi \setminus \set{a}) 
    = \Pi \setminus \set{a} \).
\end{satz}

\begin{satz}
    Zwei positive (bzw. einfache) Systeme in \( \Phi \) 
    sind konjugiert unter \(W\).
\end{satz}

\subsection{Generation von einfachen Spiegelungen}%todo

\begin{defi}[Höhe eines Vektors]
    Sei \( \Delta \) ein einfaches Wurzelsystem 
    \( \beta \in \Phi \).
    \[ \beta = \sum_{\alpha \in \Delta} c_\alpha \alpha. \]
    \( \sum c_\alpha \) wird auch die Höhe von \( \beta \) 
    (relativ zu \( \Delta \)) genannt.\\
    Wenn \( \height \beta = 1 \), so ist \( \beta \in \Delta \).
\end{defi}

\begin{satz}
    Für ein festes einfaches Wurzelsystem \( \Delta \) lässt 
    sich \( W \) durch Spiegelungen \( s_\alpha 
    (\alpha \in \Delta) \) konstruieren.
\end{satz}

\begin{kor}
    Für ein festes einfaches Wurzelsystem \( \Delta \) 
    gilt:
    \[ \forall \beta \in \Phi \; \exists w \in W: 
    w(\beta) \in \Delta \]
\end{kor}



\end{document}

