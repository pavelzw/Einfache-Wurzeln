\documentclass[12pt]{extarticle}
\usepackage[margin=2cm]{geometry}
\usepackage{../presentation/skript}

\begin{document}

\begin{defi}[\( \Delta_t^+ \) und \( \Delta_t^- \)]
    Sei \( t \in V \) so, dass \( \scalarprod{t}{r} \neq 0 
    \; \forall r \in \Delta \).

    \( \Delta \) lässt sich in zwei Teilmengen 
    \( \Delta_t^+ := 
    \set{r \in \Delta \;\vert \; \scalarprod{t}{r} > 0} \) 
    und \( \Delta_t^- := 
    \set{r \in \Delta \;\vert \; \scalarprod{t}{r} < 0} \) 
    aufteilen. 
\end{defi}

\begin{defi}[\( t \)-Basis]
    \( \Pi \subset \Delta_t^+ \) heißt \( t \)-Basis, 
    wenn gilt:
    \begin{itemize}
        \item Jedes \( r \in \Delta_t^+ \) ist 
        eine Linearkombination von Vektoren aus 
        \( \Pi \) mit ausschließlich 
        nichtnegativen Koeffizienten.
        \item \( \Pi \) ist minimal.
    \end{itemize}
    Es existiert mindestens eine \( t \)-Basis, 
    da \( \Delta \) endlich ist.
\end{defi}
\( \Pi \) sei ab jetzt immer eine \(t\)-Basis für ein 
\(t\).

\begin{defi}[\( t \)-positiv]
    \( x \in V \) heißt \(t\)-positiv, wenn 
    die Linearkombination von \(x\) zur 
    \(t\)-Basis ausschließlich nichtnegativ ist.

    Analog für \( t \)-negativ.
\end{defi}
\begin{bem}
    Hängt nicht von \( \Pi \) ab, da \( \Pi \) 
    eindeutig.
\end{bem}

% Satz 4
% nicht unbedingt beweisen
\begin{satz} % Proposition 4.1.4
    Seien \( r_i, r_j \in \Pi, i \neq j \) und 
    \( \lambda_i, \lambda_j \in \R_+ \). 
    Der Vektor \( x = \lambda_i r_i - \lambda_j r_j \) 
    ist weder \( t \)-positiv noch \( t \)-negativ.
\end{satz}
\begin{bew}
    Wenn \( x \) positiv ist, kann man schreiben 
    \[ x = \lambda_i r_i - \lambda_j r_j = \sum_{k=1}^n \mu_k r_k. \]
    mit allen \( \mu_k \geq 0 \) [da \(x \ t\)-positiv].
    
    Falls \( \lambda_i \leq \mu_i \), dann gilt 
    \[ 0 = (\mu_i - \lambda_i) r_i + (\mu_j + \lambda_j) r_j 
    + \sum \set{\mu_k r_k, k \neq i,j}. \]
    [Dann aber auch Skalarprodukt gleich \(0\)]
    \[ 0 = \scalarprod{t}{(\mu_i - \lambda_i) r_i + (\mu_j + \lambda_j) r_j 
    + \sum \set{\mu_k r_k, k \neq i,j}} \]
    [da \( \lambda_i \leq \mu_i \) und deswegen alle Koeffizienten positiv]
    \[ \geq \lambda_j \scalarprod{t}{r_j} > 0 \text{\Lightning{}}. \]
    Wenn \( \lambda_i > \mu_i \), dann 
    \[ (\lambda_i - \mu_i)r_i = (\mu_j + \lambda_j)r_j 
    + \sum \set{\mu_k r_k, k \neq i,j} \]
    \[ \Leftrightarrow r_i = 
    \frac{\mu_j + \lambda_j}{\lambda_i - \mu_i}r_j 
    + \sum \set{\frac{\mu_k}{\lambda_i - \mu_i} r_k, k \neq i,j} \]
    mit allen Koeffizienten nichtnegativ \Lightning{} zu Minimalität 
    von \( \Pi \).

    Wenn \( x \) \(t\)-negativ wäre, dann wäre \(-x\) \(t\)-positiv.
    Geht nach obiger Argumentation nicht (wenn man \( i \) und \(j\) 
    tauscht).
\end{bew}

% Satz 5
\begin{satz} % Proposition 4.1.5
    Seien \( r_i, r_j \in \Pi, i \neq j \), es 
    sei \( S_i \) die Spiegelung an \( r_i \). 

    Dann ist \( S_i r_j \in \Delta_t^+ \) und 
    \( \scalarprod{r_i}{r_j} \leq 0 \).
\end{satz}
\begin{bew}
    Da \( S_i r_j \in \Delta \), ist \( S_i r_j \) 
    entweder positiv oder negativ. Da 
    \[ S_i r_j = r_j - 2 
    \frac{\scalarprod{r_j}{r_i}}{\scalarprod{r_i}{r_i}} r_i, \]
    mit \( \lambda_j = 1 \), muss 
    \( \lambda_i \geq 0 \), da sonst 
    \( S_i r_j \notin \Delta \). 

    \( \Rightarrow \scalarprod{r_i}{r_j} \leq 0 
    \Rightarrow S_i r_j \) 
    positiv.
\end{bew}
\begin{bem}
    Geometrische Bedeutung:
    \( \scalarprod{r_i}{r_j} \leq 0 \) bedeutet, 
    dass Winkel zwischen \(r_i\) und \(r_j\) 
    stumpf ist, da Skalarprod der \( \cos \) 
    vom Winkel ist.
\end{bem}
{\Large{}[Satz 5 zeigen an Beispiel.]}

% Satz 6
\begin{satz} % Proposition 1.4.6
    Seien \( x_1, \ldots, x_m \in V \) alle auf derselben 
    Seite einer Hyperebene \( \mathscr{P} \), 
    d. h. \( \langle x_i, x \rangle > 0 \;\forall i \) für 
    ein \( x\in V \). \\
    Falls \( \scalarprod{x_i}{x_j} \leq 0 \) immer wenn 
    \( i \neq j \), dann ist \( \set{x_1, \ldots, x_m} \) 
    linear unabhängig.
\end{satz}

\begin{bew}
    Wenn es linear abhängig wäre, dann gäbe es 
    \( \lambda_i, \mu_i \geq 0 \) mit 
    manchen \( \lambda_i > 0 \), sodass 
    \[ \sum_{i=1}^k \lambda_i x_i 
    = \sum_{i=k+1}^m \mu_i x_i. \]
    
    \( \lambda_i, \mu_i \geq 0 \), da 
    \( \sum_{i=1}^m \lambda_i r_i = 0 \) 
    negative Koeffizienten nach rechts schreiben, 
    dann werden sie positiv. Bei Bedarf 
    Reihenfolge vertauschen.

    Dann gilt 
    \begin{align*}
        0 \leq \norm{\sum_{i=1}^k \lambda_i x_i}^2 
        &= \scalarprod{\sum_{i=1}^{k} \lambda_{i} x_{i}}{
        \sum_{j=1}^{k} \lambda_{j} x_{j}} \\
        &= \scalarprod{\sum_{i=1}^{k} \lambda_{i} x_{i}}{
        \sum_{j=k+1}^{m} \mu_{j} x_{j}} \\
        &= \sum_{i=1}^{k} \sum_{j=k+1}^{m} \lambda_{i} \mu_{j}
        \scalarprod{x_i}{x_j} 
        \leq 0. % nach Voraussetzung
    \end{align*}

    Da alle auf derselben Seite der Hyperebene sind, 
    gibt es ein \( x \in V: \scalarprod{x_i}{x} > 0 
    \;\forall i \in \set{1,\ldots, m} \).
    Es gilt jedoch nun 
    \[ 0 = \scalarprod{\sum_{i=1}^{k} \lambda_i x_i}{x} 
    = \sum_{i=1}^k \lambda_i \scalarprod{x_i}{x} > 0. \text{ \Lightning{}} \]
    % da manche lambda_i > 0
    Deshalb muss \( \set{x_1, \ldots, x_m} \) linear 
    unabhängig sein.
\end{bew}
{\Large{}[Satz 6 zeigen an Beispiel.]}

% Satz 7
\begin{satz} % Theorem 1.4.7
    Sei \( \Pi \) eine \(t\)-Basis von \( \Delta \). Dann ist 
    \( \Pi \) eine Basis für \( V \).
\end{satz}
\begin{bew}
    \( \Delta \) erzeugt \( V \). (s. Vortrag \gqq{Wurzelsysteme})\\
    Da jedes \( r \in \Delta \) eine Linearkombination von \( \Pi \) ist, 
    ist \( \Pi \) Erzeugendensystem.

    \( \Pi = \set{r_1, \ldots, r_m} \) alle auf derselben Seite 
    von \( \mathscr{P} \). Nach Satz 5 ist \( \scalarprod{r_i}{r_j} \leq 0 \)
    \( \forall i \neq j \). \( \oversett{Satz 6}{\Rightarrow} \Pi \) 
    ist linear unabhängig.

    \( \Rightarrow \Pi \) ist eine Basis von \(V\).
\end{bew}

% Satz 8
\begin{satz} % Proposition 4.1.8
    Es gibt nur eine \( t \)-Basis von \( \Delta \).
\end{satz}
\begin{bew}
    Es seien \( \Pi_1, \Pi_2\ t \)-Basen.\\
    Jede Wurzel in \( \Pi_1 \) ist eine nichtnegative 
    Linearkombination an Elementen in \( \Pi_2 \).\\
    \( \Rightarrow \) die Basiswechselmatrix \(A\) von 
    \( \Pi_2 \) hat nur nichtnegative Einträge.\\
    Analog hat \( B := A^{-1} \) auch nur nichtnegative 
    Einträge. \\
    Seien \( a_1, \ldots, a_n \) die Zeilen von \(A\) 
    und \( b_1, \ldots, b_n \) die Spalten von \(B\).
    Da \( AB = I_n \), gilt 
    \[ \scalarprod{a_1^T}{b_i} = 0 \;\forall i \in \set{2, \ldots, n}. \]
    (Eintrag der \( 1 \). Zeile, \( i \)-ten Spalte der Matrixmultiplikation (\( I_n \)))

    Es gibt maximal eine Zeile \( j \), für die Eintrag in allen 
    \( b_2, \ldots, b_n \) gleich Null, sonst wäre 
    \(B\) linear abhängig.

    \( \Rightarrow a_1 \) hat maximal einen 
    nichtnull Eintrag (da \(a_1\) und \( b_2, \ldots, b_n \) nichtnegativ).\\
    Analog für \( a_i, i \in \set{1, \ldots, n} \).

    Da \(A\) bijektiv ist, hat \(A\) in jeder Zeile und 
    Spalte jeweils einen \\
    nichtnegativen Eintrag.

    Sei \(r\) eine Wurzel aus \( \Pi_1 \).
    Da \( \lambda \cdot r \) nur für 
    \( \lambda = 1 \) eine Wurzel ist, 
    ist \( A \) eine Permutationsmatrix.
    \[ \Rightarrow \Pi_1 = \Pi_2. \]
\end{bew}

[Beispiele: Sagen, dass auch nur eine \(t\)-Basis 
möglich ist (wie in Satz 8).]

% Satz 9
\begin{satz} % Proposition 4.1.9
    % nicht unbedingt beweisen
    Sei \( S_i \) die Spiegelung an 
    \( r_i \in \Pi = \set{r_1, \ldots, r_n} \).
    Wenn \( r \in \Delta^+ \) und \( r_i \neq r \), 
    dann gilt \( S_i r \in \Delta^+ \).
\end{satz}
[Zeigen an Beispiel]

% Satz 10
\begin{satz} % Proposition 4.1.11
    % nicht so interessant, nur satz hinschreiben
    Sei \( r \in \Delta^+ \). Es existiert ein \( T \in 
    \mathscr{G}_t := \langle 
    \set{S_i \;\vert\; r_i \in \Pi} \rangle \), sodass 
    \( Tr \in \Pi \).
\end{satz}
[Zeigen an Beispiel]

% Satz 11
\begin{satz} % Theorem 4.1.12
    Die einfachen Spiegelungen \( S_1, \ldots, S_n \) 
    erzeugen \( \mathscr{G} \).
\end{satz}

\begin{bew}
    Da \( \mathscr{G} = \langle S_r \;\vert\; r \in \Delta \rangle \) 
    und \( S_{-r} = S_r \), genügt es, zu zeigen, dass wenn 
    \( r\in \Delta^+ \), dann \( S_r \in \mathscr{G}_t \).

    Sei \( r \in \Delta^+ \). Nach Satz 10 gibt es 
    ein \( T \in \mathscr{G}_t \), sodass 
    \( r_i := Tr \in \Pi \).\\

    Erinnerung letzter Vortrag: (Satz 2.5)\\
    \gqq{Sei \( r \in V \) Wurzel. 
    Für \( T \in \mathscr{G} \) ist \( Tr \) Wurzel von \(\mathscr{G}\).}

    \[ S_r = T^{-1} S_i T \in \mathscr{G}_t. \]
    (Abgeschlossenheit von \( \mathscr{G} \), \(T, T^{-1}, S_i \in \mathscr{G}\))
    \[ \Rightarrow \mathscr{G} \subset \mathscr{G}_t. \]
    Da \( \mathscr{G} \supset \mathscr{G}_t \) sowieso klar, gilt
    \[ \Rightarrow \mathscr{G} = \mathscr{G}_t. \]
\end{bew}

\begin{tikzpicture}[scale=3]
    \draw[thin, gray] (-1.5,0) -- (1.5,0);
    \draw[thin, gray] (0,-1.5) -- (0,1.5);
    \draw[->] (0,0) -- (0,1) node[right] {\( 1 \)};
    \draw[->] (0,0) -- (0,-1);
    \draw[->] (0,0) -- (-1,0);
    \draw[->] (0,0) -- (1,0) node[above left] {\( 1 \)};
    \draw (1,0) node[above right] {\( r_1 \)};
    \draw[->] (0,0) -- (-1, 1);
    \draw (-1, 1) node[above right] {\( r_2 \)};
    \draw[->] (0,0) -- (-1,-1);
    \draw[->] (0,0) -- (1,1);
    \draw[->] (0,0) -- (1,-1);
    \draw[->] (0,0) -- (0.77, 1.85) node[above right] {\( t \)};
    \draw[dashed] (-1.3,-0.77 / 1.85 * -1.3) -- (1.3, -0.77 / 1.85 * 1.3) 
    node[above right] {\( t^\bot \)};
\end{tikzpicture}

\end{document}