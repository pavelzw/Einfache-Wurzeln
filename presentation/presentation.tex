\documentclass[18pt,draft]{beamer}

\IfFileExists{skript.sty}{\usepackage{skript}}{\usepackage{../skript}}



\title[Einfache Wurzeln]{Endliche Spiegelungsgruppen:\\
Einfache Wurzeln}
\author{Pavel Zwerschke}
\date{17. Juni 2019}

\begin{document}

\begin{frame}
    \maketitle
\end{frame}

\begin{frame}
    \tableofcontents
\end{frame}

\section{Wiederholungen Wurzeln}

\begin{frame}{Wurzel}
    Spiegelung an Hyperebene \( \scrP \) 
    lässt sich durch einen Vektor
    \( 0 \neq r \in \scrP \)
    beschreiben:
    \[ s_r(x) = x 
    - 2\frac{\scalarprod{x}{r}}{\scalarprod{r}{r}} r \]
    Es gilt \( \forall x \in \scrP: s_r(x) = x \) 
    und \( s_r(r) = -r \).

    Da Basis \( B \subset \scrP \cup \set{r} \) 
    existiert, folgt aus linearer Fortsetzung, dass 
    \( s_r \) Spiegelung an \( \scrP \) darstellt.

    Außerdem gilt \( s_r^2 = 1 \).

    \( \Rightarrow \pm r \) sind Wurzeln von \( s_r \).

    \( \Delta = \set{ r \;\vert\; r \text{ ist Wurzel} } \) 
    nennen wir Wurzelsystem.
\end{frame}
\section{Positive und einfache Wurzeln}
\begin{frame}{Problem mit Wurzelsystemen}
    Problem: \( \Delta \) kann sehr groß werden.

    Lösungsansatz: Suchen nach linear unabhängigen 
    Untermengen von \( \Delta \), mit denen \( \Delta \)
    konstruiert werden kann.
\end{frame}

\begin{frame}{\( \Delta_t^+ \) und \( \Delta_t^- \)}
    Sei \( t \in V \) so, dass \( \scalarprod{t}{r} \neq 0 
    \forall r \in \Delta \).\\
    \( \Delta \) lässt sich in zwei Teilmengen 
    \( \Delta_t^+ = 
    \set{r \in \Delta \;\vert \; \scalarprod{t}{r} > 0} \) 
    und \( \Delta_t^- = 
    \set{r \in \Delta \;\vert \; \scalarprod{t}{r} < 0} \) 
    aufteilen.\\
    % geometrisch gesehen liegen Delta^+ und Delta^- 
    % auf unterschiedlichen Hyperebenen von 
    % t^\bot
    Sei \( r \in \Delta \). \( \Rightarrow -r \in \Delta \).
    \( \scalarprod{t}{-r} = -\scalarprod{t}{r} \).\\
    \( \Rightarrow r \) und \( -r \) sind in 
    gegensätzlichen Mengen.
    \[ \Rightarrow \abs{\Delta_t^+} = \abs{\Delta_t^-}. \]
\end{frame}

\begin{frame}{Definition \( t \)-Basis}
    \( \Pi \subset \Delta_t^+ \) heißt \( t \)-Basis, 
    wenn gilt:
    \begin{itemize}
        \item Jedes \( r \in \Delta_t^+ \) ist 
        eine Linearkombination mit ausschließlich 
        nichtnegativen Koeffizienten.
        \item \( \Pi \) ist minimal.
    \end{itemize}
    Es existiert mindestens eine \( t \)-Basis, 
    da \( \Delta \) endlich ist.\\
    Da \( \Delta_t^+ = - \Delta_t^- \), ist jedes 
    \( r \in \Delta_t^- \) eine Linearkombination 
    mit ausschließlich nichtpositiven Koeffizienten.\\
    \( \Pi \) ist eine Basis. % TODO zu beweisen?
\end{frame}

\begin{frame}{Definition \(t\)-positiv}
    \( x \in V \) heißt \(t\)-positiv, wenn 
    die Linearkombination von \(x\) zur 
    Basis ausschließlich nichtnegativ ist.\\
    Analog für \( t \)-negativ.\\
    Hängt nicht von \( \Pi \) ab, da \( \Pi \) 
    eindeutig.
\end{frame}

% todo

\begin{frame}{Ordnung im Vektorraum}
    Transitive Ordnung soll folgende Axiome erfüllen:
    \begin{enumerate}
        \item \( \forall \lambda, \mu \in V \) gilt 
        entweder \( \lambda > \mu, \lambda < \mu \) 
        oder \( \lambda = \mu \).
        \item \( \forall \lambda, \mu, \nu \in V \) 
        gilt 
        \[ \mu < \nu 
        \Rightarrow \mu + \lambda < \nu + \lambda. \]
        \item Seien \( \mu, \nu \in V, 
        c \in \R \setminus \set{0} \).
        \[ \mu < \nu \Rightarrow 
        \begin{cases}
            c \mu < c \nu, & c > 0 \\
            c \mu > c \nu, & c < 0
        \end{cases}. \]
    \end{enumerate}
\end{frame}
\begin{frame}{Ordnung im Vektorraum}
    Wie kommen wir auf so eine Ordnung?

    Wähle geordnete Basis 
    \( B = (b_1, \ldots, b_n) \). 
    Seien \( v = \sum_{i=1}^n \lambda_i b_i \)
    und \( w = \sum_{i=1}^n \mu_i b_i \).
    
    Es gelte nun
    \[ v < w \Leftrightarrow
    \lambda_k < \mu_k \text{ für kleinstes } k \text{ mit } 
    \lambda_k \neq \mu_k. \]
\end{frame}

\begin{frame}{Positives Wurzelsystem}
    \( \Pi \subset \Phi \) heißt positives 
    System, wenn es aus allen positiven Wurzeln 
    aus \( \Phi \) besteht.
    \( \Phi \) lässt sich aufteilen in 
    \( \Pi \) und \( -\Pi \), da Wurzeln als Paare 
    vorkommen.
    \[ \Pi \cap -\Pi = \emptyset, \Pi \cup -\Pi = \Phi. \]
\end{frame}


\end{document}